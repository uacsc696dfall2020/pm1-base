% 10 pts
\section{Background}
\label{sec:background}

Expand on what the reader needs to know and understand if it necessary to
adequately describe the problem. This may include definitions of terms and
explanations related to the problem area trying to be solved. For example, if
you want to evaluate a visualization system for editing film, you may need to
explain what people editing film are trying to accomplish with the system.

This section should answer the question "What does someone unfamiliar with the
area need to know to understand this project?" Often times this section
focuses on the \textit{domain}. For example, a new visualization for biology
data will explain the type of biology data used, the terminology that will be
used to describe it, the people who use it, and what they hope to learn from
it. If there is no specific domain, the background section may describe
general terms about the data abstraction (e.g. "We define a graph $G = (V, E)$
as..."), common practices, or other information for visualization researchers
who are not as familiar with this problem in comparison to all the others. 

% 10 pts
\subsection{Related Work}
\label{sec:related}

Discuss the work related to your project---include other related projects,
systems, or experiments, whether they be from visualization, other computing
areas, or even outside of computing. There are likely related academic works
(journal and conference articles), but you can also cite websites, books, and
other sources of information.

Prominent related works should be included for this milestone with a more
broad listing in the next milestone. (If you do an in depth related work
section in this milestone, you will not be expect to do ``more'' for the next
one.) 

If you are proposing a literature review, this section must include a
discussion of what similar ones exist already and should tie into the
motivation. Is the existing one old enough to be missing many relevant
advances? Are the other ones focused on related but different areas? Is the
previous one too broad and you seek a narrow focus?

For example, if my project involves comparing tools to analyze performance
data of distributed systems, I might want to cite
Perfopticon~\cite{Moritz:2015:EuroVis}. If my project involves how to evaluate
the design of a new library to support creating new visualizations, I might
want to cite d3js~\cite{d3js}. Note here I am using keys such as
"Moritz:2015:EuroVis" for citations. These refer to the full definitions in
proposal.bib.

ACM Digital Library makes it easy to get bib files for proposal.bib and there
are guides online for citing things like books~\cite{ware:2004:IVP},
theses~\cite{levoy:1989:DSV}, journal articles~\cite{Lorensen:1987:MCA}, and
conference proceedings~\cite{Nielson:1991:TAD}. There's even a format for
miscellaneous references (misc) such as websites and libraries not associated
with a article.

